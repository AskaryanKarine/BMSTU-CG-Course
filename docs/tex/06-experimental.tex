\section{Экспериментальная часть}

В данном разделе проводится сравнительный анализ реализованных алгоритмов по затрачиваемому реальному времени.

\subsection{Технические характеристики}
Технические характеристики устройства, на котором проводился эксперимент:
\begin{itemize}
	\item операционная система Windows 10 x64 Домашняя;
	\item оперативная память 15 Гб;
	\item процессор Intel(R) Core(TM) i5-10300H CPU @ 2.50 ГГц с 8 логическими ядрам~\cite{processor}.
\end{itemize}

\subsection{Замеры реального времени}
Для тестирования брались квадратные изображения размером от 128$\times$128 пикселей до 512$\times$512 с шагом 128, содержащие три одинаковых сферы. 

Класс std::chrono::system\_clock~\cite{clock} был использован для замеров затрачиваемого реализациями алгоритмов реального времени.

Для построения графиков использовалась библиотека matplotlib для языка программирования Python~\cite{mpl}.

В таблице~\ref{table:times} представлен результат замеров времени. На рисунке~\ref{img:time.pdf} приведен график, отражающий зависимость времени выполнения реализаций алгоритмов от количества потоков и линейного размера квадратного изображения.
\newpage
\begin{table}[H]
	\caption{Время выполнения реализации алгоритмов при разном количестве потоков (в мс.)}
	\label{table:times}
	\begin{center}
		\begin{tabular}{|c|cccc|}
			\hline
			\multirow{2}{*}{\begin{tabular}[c]{@{}c@{}}Количество\\ потоков\end{tabular}} & \multicolumn{4}{c|}{Линейный размер изображения}                                                             \\ \cline{2-5} 
			& \multicolumn{1}{c|}{128}    & \multicolumn{1}{c|}{256}     & \multicolumn{1}{c|}{384}     & 512     \\ \hline
			0                                                                             & \multicolumn{1}{c|}{50.441} & \multicolumn{1}{c|}{215.668} & \multicolumn{1}{c|}{546.182} & 827.216 \\ \hline
			1                                                                             & \multicolumn{1}{c|}{52.3}   & \multicolumn{1}{c|}{211.573} & \multicolumn{1}{c|}{466.08}  & 836.741 \\ \hline
			2                                                                             & \multicolumn{1}{c|}{41.559} & \multicolumn{1}{c|}{140.251} & \multicolumn{1}{c|}{316.52}  & 557.305 \\ \hline
			4                                                                             & \multicolumn{1}{c|}{26.982} & \multicolumn{1}{c|}{119.199} & \multicolumn{1}{c|}{266.242} & 436.737 \\ \hline
			8                                                                             & \multicolumn{1}{c|}{19.972} & \multicolumn{1}{c|}{75.809}  & \multicolumn{1}{c|}{170.347} & 310.915 \\ \hline
			16                                                                            & \multicolumn{1}{c|}{14.552} & \multicolumn{1}{c|}{51.035}  & \multicolumn{1}{c|}{157.271} & 229.747 \\ \hline
			32                                                                            & \multicolumn{1}{c|}{13.533} & \multicolumn{1}{c|}{53.764}  & \multicolumn{1}{c|}{113.155} & 192.211 \\ \hline
			64                                                                            & \multicolumn{1}{c|}{14.922} & \multicolumn{1}{c|}{47.623}  & \multicolumn{1}{c|}{184.554} & 184.554 \\ \hline
			128                                                                           & \multicolumn{1}{c|}{20.28}  & \multicolumn{1}{c|}{55.28}   & \multicolumn{1}{c|}{104.377} & 187.823 \\ \hline
		\end{tabular}
	\end{center}
\end{table}

\img{0.7\textwidth}{time.pdf}{Зависимость времени работы реализаций алгоритмов от количества потоков и размера изображения}

\subsection*{Выводы}
В данном разделе были проведены замеры времени работы реализаций всех трех алгоритмов. Из них можно сделать следующие выводы.

Наибольшая эффективность достигается при 64 потоках. Например, изображение в 512$\times$512 пикселей генерируется быстрее на 64 потоках в 1.7 раз, чем на 8 потоках.

При увеличении размера изображения росло время рендера. Так, при 8 потоках изображение размером 128$\times$128 синтезируется быстрее изображения размером 256$\times$256 в 3.8 раз, так как необходимо обработать больше пикселей. 


\newpage

