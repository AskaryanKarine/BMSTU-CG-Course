%\pagenumbering{arabic}
\section*{Введение}
\phantomsection
\addcontentsline{toc}{section}{Введение}
Компьютерная графика является неотъемлемой частью жизни человека. Она обладает высоким спросом: её используют сфере развлечений, например, эффекты в кинофильмах и сериалах, анимации в играх; в научных и инженерных дисциплинах, каких как, медицина, геофизика, ядерная физика, картография, для визуализации и лучшего восприятия информации. Компьютерную графику так же можно использовать в школе на уроках физики, информатики, математики для демонстрации различных явлений.

Прогресс компьютерной графики не стоит на месте: со временем развиваются и оптимизируются алгоритмы, которые позволяют получить реалистичное трехмерное изображение. Вместе с прогрессом растут и требования к реалистичности, что приводит к росту как сложности алгоритмов, так и к затратам ресурсов компьютера, таких как оперативная память и время работы алгоритма. 

\textbf{Целью} курсовой работы является разработка программного обеспечения визуализации геометрических объектов, расположенных за полупрозрачной сферой. 


Для достижения поставленной цели необходимо решить следующие \textbf{задачи}:
\begin{enumerate}
	\item формализовать объекты сцены;
	\item выбрать и описать алгоритмы реализации, необходимые для визуализации поставленной задачи;
	\item составить требования к программному продукту;
	\item реализовать выбранные алгоритмы;
	\item исследовать время работы реализации алгоритма от количества потоков.
\end{enumerate}


\newpage