\section{Конструкторская часть}
В данном разделе описываются математические основы метода моделирования и разрабатываются схемы алгоритмов на основе теоретических данных из аналитического раздела.

\subsection{Математические основы для реализации алгоритма обратной трассировки лучей}

\subsubsection{Определение направлений отраженного и преломленного лучей}
На рисунке~\ref{img:rays.pdf} представлены направления отраженного и преломленного лучей.
\img{0.5\textwidth}{rays.pdf}{Направления лучей}
$\overrightarrow{OA_1}$ -- проекция $\overrightarrow{AO}$. Тогда, $AO = ON(ON\cdot(-AO))$ и $A_1A = -AO-OA_1 = -AO+ON(ON\cdot AO)$.
Тогда $\overrightarrow{OB})$:
\begin{equation}
	OB = -OA -2 ON(ON\cdot AO).
\end{equation}

Пусть $\eta_{1}$ и $\eta_{2}$ показатели преломления сред 1 и 2, соответственно. Применяя закон Снеллиуса можно вычислить преломленный луч:
\begin{equation}
	OC = \frac{\eta_{1}}{\eta_{2}}AO + (\frac{\eta_{1}}{\eta_{2}} cos(\alpha_1) - cos(\alpha_2)) ON,
\end{equation}
где $cos(\alpha_2) = \sqrt{1 -   (\frac{\eta_{1}}{\eta_{2}}) ^ 2 (1 - cos(\alpha_1))^2  }$.


\subsubsection{Расчет нормали и коэффициентов плоскости по точкам}
Уравнение вида 
\begin{equation}
	Ax + By + Cz + D = 0
\end{equation}
называют общим уравнением плоскости~\cite{angeom}.

Если есть три точки, не лежащие на одной прямой, тогда существует единственная плоскость, которой эти точки принадлежат. Пусть $M(x,y,z)$ -- произвольная точка на плоскости, а точки $M_i(x_i, y_i, z_i), i = 1, 2, 3$ нам известны. Тогда для описания плоскости как множества точек $M$ векторы $\overrightarrow{M_1M_2}, \overrightarrow{M_1M_3}, \overrightarrow{M_1M}$ должны быть компланарны. Записав критерий компланарности векторов, получается следующий вид:
\begin{equation}
	\label{planeMat}
	\begin{vmatrix}
		x - x_1 & x_2 - x_1 & x_3 - x_1 \\
		y - y_1 & y_2 - y_1 & y_3 - y_1 \\
		z - z_1 & z_2 - z_1 & z_3 - z_1
	\end{vmatrix} = 0.
\end{equation}
Тогда из формулы (\ref{planeMat}) будут получены следующие коэффициенты плоскости:
\begin{gather}
	A = (y_2 - y_1)(z_3 - z_1) - (y_3 - y_1)(z_2 - z_1),\\
	B = (x_2 - x_1)(z_3 - z_1) - (x_3 - x_1)(z_2 - z_1),\\
	C = (x_2 = x_1)(y_3 - y_1) - (x_3 - x_1)(y_2 - y_1),\\
	D = -Ax_1 + By_1 - Cz_1.
\end{gather}

Если после подстановки полученных коэффициентов получается положительное значение, то надо умножить их на $-1$. Вектор нормали к данному полигону можно представить как $n = \begin{pmatrix} A & B & C \end{pmatrix}^T$. 

\subsubsection{Поиск пересечения луча со сферой}
Луч определяется своим положением $p$ и единичным вектором направления $u$. Тогда, луч~\cite{ray_sphere} --- это множество точек 
\begin{equation}
	\{p + \alpha u |\alpha \geq 0\}.
\end{equation}

Сфера задается точкой центра $c$ и радиусом $r > 0$. Прежде всего находится точка $q$ -- ближайшая к центру сферы на луче $p$, следующим образом:
\begin{equation}
	\label{sph2}
	q = p + \alpha u,
\end{equation} 
где $\alpha$ -- расстояние от $q$ до $p$.
На рисунке~\ref{img:raySphere.pdf} представлены все обозначения для наглядности.
\img{0.45\textwidth}{raySphere.pdf}{Геометрические построения}
Из построения получается:
\begin{equation}
	\label{sph1}
	(q-c) \perp u \Rightarrow 0 = (q-c) \cdot u = (p+\alpha u)u.
\end{equation}

Т.к. $u \cdot u = 1$, из правой части формулы~(\ref{sph1}) можно выразить $\alpha$:
\begin{equation}
	\label{sph3}
	\alpha = -(p - c) \cdot u.
\end{equation}
Подставляя (\ref{sph3}) в (\ref{sph2}) можно выразить точку $q$:
\begin{equation}
	q = p - ((p - c)\cdot u ) \cdot u.
\end{equation}

Когда точка $q$ найдена, можно проверить пересекает луч сферу или нет:
\begin{equation}
	|q-c|^2\leq r^2.
\end{equation}
Если $q$ лежит внутри сферы, то обозначим за $b = |q - c|$ и $a = \sqrt{r^2 - b^2}$. В общем случае луч пересекает сферу в двух точках:
\begin{equation}
	q_{1,2} = p + (\alpha \pm a)\cdot u.
\end{equation}
Если $\alpha \geq a$, то луч пересекается со сферой в точке $q_1$. Иначе, необходимо проверить $\alpha + a > 0$. Если это неравенство выполняется, то точка лежит внутри сферы, а точка $q_2$ -- точка падения луча на внутреннюю поверхность сферы~\cite{ray_sphere}.


\subsection{Пересечение луча с многогранником}
Есть ситуации, когда можно сразу понять, что многогранник не пересекается лучом, для этого необходимо поместить многогранник в сферическую оболочку и проверить пересечение со сферой по описанному выше алгоритму. В ином случае необходимо осуществить поиск пересечений со всеми гранями многогранника.

Пусть для поиска пересечения луча с гранью многогранника плоскость $\Gamma$ задается вектором нормали $n$ и числом $d$. Точки $x$ плоскости удовлетворяют уравнению $x \cdot n = d$~\cite{ray_sphere}.
Луч задан точкой положения $p$ и единичным вектором направления $u$. Пусть 
\begin{equation}
	q = p + \alpha u, \alpha \in \mathbb{R}.
\end{equation}

Тогда
\begin{equation}
	\label{plane1}
	q \in \Gamma \Leftrightarrow d = q \cdot n = p\cdot n + \alpha u\cdot n. 
\end{equation}
Из формулы (\ref{plane1}) получается следующее:
\begin{equation}
	\alpha = \frac{d - p \cdot n}{u \cdot n}.
\end{equation}

Так как плоскость разбивает пространство на два полупространства, пусть полупространство, куда направлена нормаль $n$ будет называться <<над>> плоскостью, а вторая часть -- <<под>> плоскостью.

Тогда,
\begin{equation}
	d - p\cdot n \begin{cases}
		< 0,\qquad \text{точка $p$ находится <<над>> плоскостью},\\ 
		> 0,\qquad \text{точка $p$ находится <<под>> плоскостью},\\
		= 0,\qquad \text{луч параллелен плоскости}.
	\end{cases}
\end{equation}

Расстояние между $p$ и $q$ обозначается как $\alpha$. Если $\alpha < 0$, то луч не пересекает плоскость.

\subsection{Проектирование алгоритма обратной трассировки лучей}
На рисунке~\ref{img:schemes-cast_ray.pdf} представлена схема работы алгоритма обратной трассировки лучей. 
\img{1.1\textwidth}{schemes-cast_ray.pdf}{Схема работы алгоритма обратной трассировки лучей}

\subsection*{Выводы}
В данном разделе были описаны математические основы метода моделирования и разработаны схемы алгоритмов на основе теоретических данных из аналитического раздела.


\newpage