\section{Аналитическая часть}
В данном разделе выполняется формализация объектов сцены, а также приводятся существующие методы и алгоритмы для решения данной задачи.

\subsection{Формализация объектов сцены}
Сцена состоит из следующих объектов:
\begin{enumerate}
	\item прозрачная сфера --- видимый объект, который задается аналитически, коэффициенты преломления и прозрачности задаются пользователем;
	\item геометрические объекты --- видимые объекты из ограниченного набора, представленного ниже, каждый из которых характеризуется своими параметрами и цветом, который задает пользователь;
	\item плоскость, параллельная координатной плоскости $Oyz$ --- видимый объект, ограничивающий снизу положение других объектов на сцене;
	\item камера --- невидимый точечный объект, с помощью которого рассчитывается перспективное отображение объектов сцены;
	\item источник света -- невидимый точечный объект, который задается тремя координатами положения.
\end{enumerate}

В ограниченный набор объектов входят следующие виды геометрических примитивов:
\begin{enumerate}
	\item шар;
	\item треугольная и четырехугольная пирамиды;
	\item параллелепипед;
	\item от треугольной до восьмиугольной призмы.
\end{enumerate}

Шар характеризуется координатами центра и радиусом. Остальные объекты --- набором точек и связей между ними.


\subsection{Обзор и анализ способов задания трехмерной модели}
Модель --- отображение форм и размеров объекта, чье основное предназначение -- правильно отобразить форму и размер конкретного объекта~\cite{Kurov}. 

Используется три вида модели: каркасная, поверхностная и твердотельная.

\textbf{Каркасная модель.} Простейший вид модели, состоит из вершин и ребер, однако она может некорректно передать представление о форме и размерах объекта.

\textbf{Поверхностная модель.} Поверхность в такой модели может задаваться как аналитически, так и другим способом, например как поверхности того или иного вида с помощью, например, полигональной аппроксимации. Ее недостатком является отсутствие информации о том, с какой стороны находится материал, а с какой -- пустота.

\textbf{Твердотельная модель.} Данная модель отличается от поверхностной тем, что включает информацию о том, с какой стороны находится материал. Чаще всего для этого используется направление внутренней нормали.

\subsection{Обзор алгоритмов удаления невидимых ребер и поверхностей}
Ниже рассмотрены возможные алгоритмы построения изображений.

\textbf{Алгоритм Варнока}

Данный алгоритм работает в пространстве изображений. В рамках этого алгоритма рассматривается окно и решается пусто оно или его содержимое достаточно просто для визуализации. Если это не так, то окно разбивается на фрагменты до тех пор, пока содержимое подокна не станет достаточно простым для визуализации или пока его размер не достигнет области в один пиксель~\cite{rodgers}.

Особенностями этого алгоритма является возможность устранения лестничного эффекта и рекурсивный вызов.

\textbf{Алгоритм, использующий $Z$-буфер}

Данный алгоритм работает в пространстве изображений, используя буфер кадра для заполнения интенсивности каждого пикселя, здесь вводится некоторый $Z$-буфер (отдельный буфер глубины каждого пикселя). 
В процессе работы глубина каждого нового пикселя, который нужно занести в буфер кадра, сравнивается с глубиной пикселя, занесенного в $Z$-буфер. Если сравнение показывает, что новый пиксель расположен ближе к наблюдателю, то новое значение $z$ заносится в буфер и корректируется значение интенсивности~\cite{rodgers}.

Данный алгоритм прост в реализации, но его недостатком является большой объем требуемой памяти.

Более формальное описание алгоритма $Z$-буфера~\cite{rodgers}.
\begin{enumerate}
	\item Заполнить буфер кадра фоновым значением цвета (интенсивности).
	\item Заполнить $Z$-буфер минимальным значением.
	\item Преобразовать каждый многоугольник сцены в растровую форму.
	\item Для каждого пикселя$(x,y)$ в многоугольнике вычислить глубину $z(x,y)$.
	\item Сравнить глубину $z(x,y)$ со значением $Z$-буфер$(x,y)$.
	\item Если $z(x,y) > Z$=буфер$(x,y)$, то записать цвет (интенсивность) в буфер кадра и заменить значение в $Z$-буфере.
\end{enumerate}

\textbf{Алгоритм Робертса}

Данный алгоритм работает в объектном пространстве. Требуется, чтобы все изображаемые тела были выпуклыми, иначе они должны быть разбиты на выпуклые части. 
Тело должно представляться набором пересекающихся плоскостей. 
Основные этапы алгоритма:
\begin{enumerate}
	\item подготовка исходных данных;
	\item удаление линий, которые экранируются самим телом;
	\item удаление линий, которые экранируются другими телами;
	\item удаление тех линий пересечения тел, которые экранируются непосредственно самими телами, связанных отношением протыкания
\end{enumerate}

Особенностями этого алгоритма является высокая точность вычислений, но из-за трудоемкости этих вычислений он не может являться достаточно эффективным.

\textbf{Алгоритм прямой трассировки лучей}

Главная идея этого метода заключается в том, что наблюдатель видит любой объект посредством испускаемого неким источником света, который падает на этот объект и затем каким-то путем доходит до наблюдателя. Свет может достичь наблюдателя, отразившись от поверхности, преломившись или пройдя через нее. Если проследить за лучами света, выпущенным источником, то можно убедиться, что немного из них дойдут до наблюдателя. Данный процесс вычислительно неэффективен~\cite{rodgers}. В связи с этим он будет исключен из рассмотрения.

\textbf{Алгоритм обратной трассировки лучей}

Данный алгоритм работает в пространстве изображения. В его основе лежит принцип обратимости световых лучей, то есть вместо просчёта всех лучей испускаемых из источников сцены, можно изначально испускать первичные лучи из камеры, что значительно повышает эффективность визуализации. При попадании первичного луча в объект вычисляется цвет точки попадания. Для этого создаются вторичные лучи в зависимости от типа поверхности объекта~\cite{rodgers}.

Считается, что камера (наблюдатель) находится на положительной части оси $Oz$, а картинная плоскость ей перпендикулярна.
Тогда основные этапы алгоритма будут следующими:
\begin{enumerate}
	\item в каждый пиксель растра выпускаются лучи из положения наблюдателя;
	\item для каждого луча отслеживается траектория для определения ближайшего пересечения с объектом;
	\item для определения цвета пикселя из найденной точки пересечения испускаются лучи к каждому источнику света. Если на пути к источнику света луч пересекает другой объект, то свет от данного источника света не учитывается в расчете цвета. Если для луча от наблюдателя не найдено объектов пересечения, то пиксель закрашивается цветом фона;
	\item для расчета преломлений и отражений используются физические законы, такие как равенство угла падения и отражения и закон Снеллиуса. Найденная точка пересечения становится новой точкой наблюдения, и алгоритм испускания лучений повторяется до достижения максимальной глубины рекурсии. 
\end{enumerate}

На рисунке~\ref{img:raytracing.png} представлена визуализация алгоритма обратной трассировки лучей.
\img{0.7\textwidth}{raytracing.png}{Визуализация алгоритма обратной трассировки}

Особенностями данного алгоритма является устранение ступенчатости, а также учет эффектов отражения одного объекта от поверхности другого, преломления, прозрачности и затемнения~\cite{rodgers}.

\subsection{Обзор моделей освещения}
Модели освещения разделяются на глобальные и локальные.

Локальная модель освещения является самой простой, она рассматривает расчет освещенности самих объектов, не учитывая процессы светового взаимодействия между объектами, кроме этого взаимодействие ограничивается однократным отражением света от непрозрачной поверхности.

Глобальная модель освещения рассматривает трехмерную сцену как единую систему, описывая освещение с учетом взаимного влияния объектов, а так же учитывая многократное отражение и преломление света.

\textbf{Модель освещения Ламберта}

Модель освещения Ламберта моделирует идеальное диффузное отражение от поверхности. Это означает, что свет при попадании на поверхность рассеивается равномерно во все стороны с одинаковой интенсивностью~\cite{rodgers}.

Интенсивность освещенности в этой модели находится по следующему закону:
\begin{equation}
	I_{\alpha} = I_0\cdot cos(\alpha),	
\end{equation}
где $I_{\alpha}$ -- уровень освещенности в рассматриваемой точке, $I_0$ -- максимальный уровень освещенности, $\alpha$ -- угол между вектором нормали к плоскости и вектором, направленным от рассматриваемой точки к источнику света.

Данная модель одна из самых простых моделей освещения, она часто используется в комбинации других моделей, т.к. практически в любой другой модели освещения можно выделить диффузную составляющую.

\textbf{Модель освещения Фонга}

Модель Фонга (глобальная модель) – классическая модель освещения, которая представляет собой комбинацию диффузной составляющей (модели Ламберта) и зеркальной составляющей и работает таким образом, что кроме равномерного освещения на материале может еще появляться блик. Местонахождение блика на объекте, освещенном по модели Фонга, определяется из закона равенства углов падения и отражения~\cite{Kurov}.
Ниже представлено более подробное описание составляющих модели.
\begin{enumerate}
	\item $Ambient$ (фоновое освещение) --- константа, которая присутствует в любом участке сцены и не зависит от пространственных координат освещаемой точки и источника.
	\item $Diffuse$ (рассеянный свет) --- рассчитывается по закону косинусов (закону Ламберта):
	\begin{equation}
		\label{svet_a}
		I_d = k_d\cdot cos(L, N)\cdot i_d = k_d\cdot(L\cdot N)\cdot i_d,
	\end{equation}
	где $I_d$ -- рассеянная составляющая освещенности в точке, $k_d$ -- свойство материала воспринимать рассеянное освещение, $i_d$ -- мощность рассеянного освещения, $L$ -- направление из точки на источник, $N$ -- вектор нормали в точке.
	\item $Specular$ (зеркальная составляющая) -- зависит от того, насколько близко находится вектор отраженного луча и вектор к наблюдателю.
	
	Падающий, отраженный лучи и нормально к отражающей поверхности лежат в одной плоскости. Нормаль делит угол между лучами на две равные части. Получается, что отраженная составляющая зависит от угла между направлением на наблюдателя и отраженным лучом, что отражено в формуле (\ref{svet}).
	
	\begin{equation}
		\label{svet}
		I_s = k_s\cdot cos^{\alpha}(R, V)\cdot i_s = k_s\cdot(R\cdot V)^{\alpha}\cdot i_s,
	\end{equation}
	где $I_s$ -- зеркальная составляющая освещенности в точке, $k_s$ -- коэффициент зеркального отражение, $i_s$ -- мощность зеркального освещения, $R$ -- направление отраженного лука, $V$ -- направление на наблюдателя, $\alpha$ -- коэффициент блеска, свойство материала.
\end{enumerate}

Тогда, объединяя формулы (\ref{svet_a}) и (\ref{svet}) и учитывая наличие на сцене других тел, получатся следующее:
\begin{equation}
	I = k_{\alpha}\cdot I_{\alpha} + k_d\cdot \sum_{j}I_j(N\cdot L_j) + k_s\cdot\sum_{j}I_j(V\cdot R_j)^{\alpha} + k_s\cdot I_s + k_r\cdot I_r,
\end{equation}
где 
\begin{itemize}
	\item $k_{\alpha}$ -- коэффициент фонового освещения,
	\item $k_d$ -- коэффициент диффузного отражения,
	\item $k_s$ -- коэффициент зеркального отражения,
	\item $k_r$ -- коэффициент пропускания,
	\item $N$ -- единичный вектор нормали к поверхности,
	\item $L_j$ -- единичный вектор, направленный к $j$-му источнику света,
	\item $V$ -- единичный вектор, направленный на наблюдателя,
	\item $R_j$ -- вектор направления отраженного луча $L_j$, 
	\item $I_{\alpha}$ -- интенсивность фонового освещения, 
	\item $I_j$ -- интенсивность $j$-го источника света, 
	\item $I_s$ -- интенсивность от зеркально отраженного луча, 
	\item $I_r$ -- интенсивность от преломленного луча.
\end{itemize}

На рисунке~\ref{img:fong.jpg} изображен пример работы модели Фонга.
\img{0.27\textwidth}{fong.jpg}{Пример работы модели Фонга~\cite{fong}}

\subsection{Задача учета прозрачности объекта}

В основных моделях освещения и алгоритмах удаления невидимых линий и поверхностей рассматриваются только непрозрачные поверхности и объекты. Однако есть и прозрачные объекты, пропускающие свет. 

При переходе из одной среды в другую световой луч преломляется. Преломление рассчитывается по закону Снеллиуса:
\begin{equation}
	\eta_{1}\cdot sin(\theta) = \eta_{2} \cdot sin(\theta^{'})
\end{equation}
где $\eta_{1}$ и $\eta_{2}$ -- показатели преломления двух сред, $\theta$ -- угол падения, $\theta^{'}$ -- угол преломления.

Ни одно вещество не пропускает весь падающий свет, его часть всегда отражается, что видно на рисунке~\ref{img:snellius.pdf}
\img{0.5\textwidth}{snellius.pdf}{Геометрия преломления}

Простое преломление пропускание света можно встроить в любой алгоритм удаления невидимых поверхностей, кроме алгоритма с $Z$-буфером.
Прозрачные многоугольники или поверхности поменяются и, если видимая грань прозрачна, то в буфер кадра записывается линейная комбинация двух ближайших поверхностей. При этом интенсивность будет равна:
\begin{equation}
	I = t\cdot I_1 + (1 - t)\cdot I_2 \qquad 0 \leq t \leq 1,
\end{equation}
где $I_1$ -- видимая поверхность, $I_2$ -- поверхность, расположенная непосредственно за ней, $t$ -- коэффициент прозрачности $I_1$~\cite{rodgers}. 

При $t = 0$  поверхность абсолютно прозрачна, при $t = 1$ непрозрачна.

Для того чтобы включить преломление в модель освещения, нужно при построении видимых поверхностей учитывать не только падающий, но и отраженный и пропущенный свет. Эффективнее всего это выполняется с помощью глобальной модели освещения в сочетании с алгоритмом трассировки лучей~\cite{rodgers}.

\subsection{Сравнение рассмотренных алгоритмов}

Для решения поставленной задачи выбор был сделан в пользу твердотельной модели, так как каркасная модель не предоставляет полную информацию об объекте, а также необходимо знать с какой стороны объекта находится материал.

В таблице~\ref{table:algs} представлено сравнение алгоритмов, описанных ранее. Представленная в таблице оценка эффективности по времени была взята из~\cite{Kurov} и~\cite{rodgers} источников. 
\begin{table}[h!]
	\begin{center}
		\caption{Сравнение алгоритмов удаления ребер и невидимых поверхностей}
		\label{table:algs}
		\begin{tabular}{|c|cccc|}
			\hline
			\multirow{2}{*}{Характеристика}                                                           & \multicolumn{4}{c|}{\begin{tabular}[c]{@{}c@{}}Алгоритм \\ или класс алгоритмов\end{tabular}}                                                \\ \cline{2-5} 
			& \multicolumn{1}{c|}{В}                    & \multicolumn{1}{c|}{ЗБ}                   & \multicolumn{1}{c|}{Р}        & ОТ                   \\ \hline
			\begin{tabular}[c]{@{}c@{}}Теоретическая оценка\\ эффективности по\\ времени\end{tabular} & \multicolumn{1}{c|}{$O(w\cdot h\cdot n)$} & \multicolumn{1}{c|}{$O(w\cdot h\cdot n)$} & \multicolumn{1}{c|}{$O(n^2)$} & $O(w\cdot h\cdot n)$ \\ \hline
			\begin{tabular}[c]{@{}c@{}}Возможность построения\\ сложных сцен\end{tabular}             & \multicolumn{1}{c|}{Да}                   & \multicolumn{1}{c|}{Да}                   & \multicolumn{1}{c|}{Нет}      & Да                   \\ \hline
			\begin{tabular}[c]{@{}c@{}}Учет всех оптических \\ эффектов\end{tabular}                  & \multicolumn{1}{c|}{Нет}                  & \multicolumn{1}{c|}{Нет}                  & \multicolumn{1}{c|}{Нет}      & Да                   \\ \hline
			\begin{tabular}[c]{@{}c@{}}Наличие сложных\\ объемных вычислений\end{tabular}             & \multicolumn{1}{c|}{Нет}                  & \multicolumn{1}{c|}{Нет}                  & \multicolumn{1}{c|}{Да}       & Нет                  \\ \hline
		\end{tabular}
	\end{center}
\end{table}

В данной таблице были приняты следующие обозначения: <<В>> -- различные варианты алгоритма Варнока, <<ЗБ>> -- алгоритмы, использующие $Z$-буфер, <<Р>> -- алгоритм Робертса,  <<ОТ>> -- алгоритм обратной трассировки лучей, $w$ и $h$ -- ширина и высота изображения растра, соответственно, $n$ -- количество объектов на сцене.

Было принято решение использовать алгоритм обратной трассировки лучей, так как именно он удовлетворяет всем поставленным задачам.

\subsection*{Вывод}

В данном разделе была выполнена формализация объектов сцены и приведены существующие методы и алгоритмы для решения поставленной задачи. 
Для задания трехмерной модели была выбрана твердотельная модель, а для реализации синтеза изображения --- алгоритм обратной трассировки лучей.

\newpage