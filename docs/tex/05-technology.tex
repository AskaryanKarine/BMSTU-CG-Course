\section{Технологическая часть}
В данном разделе обосновывается выбор средств реализации и проводится тестирование.
%приводится реализация алгоритмов, схемы которых были представлены в конструкторском разделе, а также

\subsection{Выбор технологии программирования}
Структурный и объектно-ориентированный подходы являются основными технологиями в программировании. Так как данная задача представляет собой сложную систему, было принято решение об использовании объектно-ориентированного подхода. Кроме того, в выбранном подходе есть принципы инкапсуляции, полиморфизма и наследования, которые позволяют создавать более гибкие и широко расширяемые системы в отличие от структурного подхода.

\subsection{Средства реализации}
При написании программного продукта было решено задействовать язык С++~\cite{cpp}. Этот выбор обусловлен следующими факторами:
\begin{enumerate}
	\item высокая вычислительная производительность, которая позволяет сократить время синтеза изображения;
	\item поддержка объектно-ориентированного программирования, что позволяет пользоваться шаблонами и паттернами проектирования;
	\item доступность – существует большое количество учебной литературы и статей по С++ на различных языках;
	\item по сравнению с С, в С++ больше готового функционала.
\end{enumerate}
Средой разработки была выбрана среда QtCreator~\cite{qt}, что обусловлено следующими причинами:
\begin{enumerate}
	\item поддержка расширения QtDisign, которая позволяет создавать интерфейс;
	\item отладчик, позволяющий выявлять ошибки в коде программного продукта;
	\item хорошая совместимость с С++, так как фреймворк Qt написан на этом языке программирования.
\end{enumerate}


\subsection{Примеры работы программы}
На рисунке~\ref{img:interface1.png} представлен интерфейс ПО. 
\img{0.5\textwidth}{interface1.png}{Интерфейс ПО}



\subsection*{Выводы}
В данном разделе были реализованы алгоритмы линейного и параллельного конвейеров и выполнено тестирование этих реализаций.



\newpage
	