\section{Технологическая часть}
В данном разделе выдвигаются требования к ПО, обосновывается выбор средств реализации, описывается интерфейс и приводятся примеры работы программы.
%приводится реализация алгоритмов, схемы которых были представлены в конструкторском разделе, а также

\subsection{Выбор технологии программирования}
Структурный и объектно-ориентированный подходы являются основными технологиями в программировании. Так как данная задача представляет собой сложную систему, было принято решение об использовании объектно-ориентированного подхода. Кроме того, в выбранном подходе есть принципы инкапсуляции, полиморфизма и наследования, которые позволяют создавать более гибкие и широко расширяемые системы в отличие от структурного подхода.

\subsection{Требования к ПО}
Программа должна обеспечивать следующие возможности:
\begin{enumerate}
	\item изменение свойств полупрозрачной сферы, таких как коэффициенты преломления и прозрачности, цвет;
	\item добавление на сцену объект из следующего списка --- пирамида, параллелепипед, шар, призма (от 3-угольной до 8-угольной);
	\item добавление на сцену белых точечных источников света;
	\item изменение расположение источников света на сцене;
	\item изменение расположения добавленных геометрических объектов на сцене;
	\item перемещение камеры по сцене.
\end{enumerate}


Программа должна:
\begin{enumerate}
	\item корректно реагировать на действия пользователя;
	\item сохранять изображение в формате $.png$.
\end{enumerate}

\newpage
\subsection{Средства реализации}
При написании программного продукта было решено задействовать язык С++~\cite{cpp}. Этот выбор обусловлен следующими факторами:
\begin{enumerate}
	\item высокая вычислительная производительность, которая позволяет сократить время синтеза изображения;
	\item поддержка объектно-ориентированного программирования, что позволяет пользоваться шаблонами и паттернами проектирования;
	\item доступность – существует большое количество учебной литературы и статей по С++ на различных языках;
	\item по сравнению с С, в С++ больше готового функционала.
\end{enumerate}
Средой разработки была выбрана среда QtCreator~\cite{qt}, что обусловлено следующими причинами:
\begin{enumerate}
	\item поддержка расширения QtDisign, которая позволяет создавать интерфейс;
	\item отладчик, позволяющий выявлять ошибки в коде программного продукта;
	\item хорошая совместимость с С++, так как фреймворк Qt написан на этом языке программирования.
\end{enumerate}
\newpage

\subsection{Описание структуры программы}
На рисунке~\ref{img:classes.pdf} представлена диаграмма классов программы.
\img{1\textwidth}{classes.pdf}{Диаграмма классов}

\newpage
\subsection{Интерфейс программы}
На рисунке~\ref{img:interface1.png} представлен интерфейс ПО. 
\img{0.5\textwidth}{interface1.png}{Интерфейс ПО}

Далее будут рассмотрены элементы интерфейса более подробно.

На рисунке~\ref{img:interface2.png} представлена левая панель интерфейса.
На ней использованы следующие обозначения:
\begin{enumerate}
	\item выпадающий список с типами объектов;
	\item добавление объекта выбранного типа на сцену;
	\item выпадающий список со всеми объектами на сцене;
	\item удаление текущей фигуры, указанной в пункте 3;
	\item кнопка, вызывающая диалоговое окно для выбора цвета фигуры;
	\item отображение текущего выбранного цвета для новых фигур и фигуры, указанной в пункте 3;
	\item интерфейс для перемещения объекта, выбранного в пункте 3, на сцене;
	\item интерфейс для поворота вокруг своей оси объекта из пункта 3;
	\item интерфейс для изменения размера объекта, указанного в пункте 3;
	\item выпадающий список со всеми источниками света на сцене; 
	\item добавление источника света на сцену (по умолчанию появляется в точке $(0, 0, 0)$);
	\item удаление источника света, выбранного в пункте 10;
	\item интерфейс для перемещения указанного в пункте 10 источника света;
	\item кнопка, отображающая диалоговое окно для выбора цвета фона;
	\item отображение текущего фонового цвета;
	\item применение всех изменений, указанных в пунктах 5 -- 9 и 13 -- 15, позволяет редактировать несколько объектов, прежде чем рисовать их;
	\item синтез изображения.
\end{enumerate}
\img{1\textwidth}{interface2.png}{Левая панель}
\newpage
На рисунке~\ref{img:left3.png} представлено диалоговое окно выбора цвета.
\img{0.7\textwidth}{left3.png}{Диалоговое окно выбора цвета}

На рисунке~\ref{img:bottom.png} представлена нижняя панель интерфейса, на которой использованы следующие обозначения:
\begin{enumerate}
	\item группа кнопок для перемещения камеры;
	\item выбор цвета сферы с помощью диалогового окна, представленного на рисунке~\ref{img:left3.png};
	\item изменение коэффициента прозрачности, может быть в пределах от 0 до 1 включительно;
	\item изменение коэффициента преломления;
	\item вывод информации о положении объектов, указанных в выпадающих списках, и камере.
\end{enumerate}

\img{0.1\textwidth}{bottom.png}{Нижняя панель}

На рисунке~\ref{img:up.png} представлена верхняя часть интерфейса, на которой есть:
\begin{itemize}
	\item выпадающее меню "Сцена" с пунктом меню "Сохранить изображение"
	\item выход из программы
\end{itemize}

\img{0.3\textwidth}{up.png}{Верхняя панель}

Остальное пространство интерфейса занимает область для показа изображения.

\subsection{Примеры работы программы}
На рисунках~\ref{img:ex1.png} --~\ref{img:ex7.png} представлены некоторые сцены, отражающие возможности программного продукта.
\img{0.5\textwidth}{ex1.png}{Сцена, содержащая различные геометрические объекты}
\img{0.5\textwidth}{ex2.png}{Демонстрация возможности менять цвета фона, сферы и объектов, количество источников света}
\img{0.5\textwidth}{ex3.png}{Демонстрация прозрачности сферы, коэффициент $0.5$}
\img{0.5\textwidth}{ex4.png}{Демонстрация прозрачности сферы, коэффициент $0.0$}
\img{0.5\textwidth}{ex5.png}{Демонстрация прозрачности сферы, коэффициент $0.8$}
\img{0.5\textwidth}{ex6.png}{Демонстрация прозрачности сферы, коэффициент $1.0$}
\img{0.5\textwidth}{ex7.png}{Демонстрация возможности изменения положения камеры и коэффициента преломления (равен $1.37$)}

\subsection*{Выводы}
В данном разделе были выдвинуты требования к ПО, обоснован выбор средств реализации, описан интерфейс и приведены примеры работы программы.

\newpage
	