\documentclass[a4paper,14pt]{extarticle}

\usepackage{cmap} % Улучшенный поиск русских слов в полученном pdf-файле
\usepackage[T2A]{fontenc} % Поддержка русских букв
\usepackage[utf8]{inputenc} % Кодировка utf8
\usepackage[english,russian]{babel} % Языки: русский, английский

\usepackage{enumitem} % Настройка оформления списков
\setlist{nolistsep} 
\setenumerate[1]{label={\arabic*)}}

\usepackage{enumitem}


\usepackage[14pt]{extsizes} % Задание 14-размера шрифта

\usepackage{caption} % Подпись картинок и таблиц
\captionsetup{labelsep=endash} % Разделитель между номером и текстом краткое тире и пробел
\captionsetup[figure]{name={Рисунок}} % Изменяет имя для всех фигур на "Рисунок"

\usepackage{amsmath} % Что-то связанное с математикой
\usepackage{amssymb}

\usepackage[left=3cm,right=1.5cm,top=2cm,bottom=2cm]{geometry} % Задание геометрии листа

\usepackage{titlesec} % Оформление заголовков
\titleformat{\section}[block]
{\bfseries\large}
{\thesection}
{1em}
{}

\titleformat{\subsection}[hang]
{\bfseries\normalsize}
{\thesubsection}
{1em}{}

\titleformat{\subsubsection}[hang]
{\bfseries\normalsize}
{\thesubsubsection}
{1em}{}

\usepackage{setspace}
\onehalfspacing % Полуторный интервал

\frenchspacing
\usepackage{indentfirst} % Красная строка

\usepackage{listings} % Оформление листингов
\usepackage{xcolor} % Добавление цветов 

\usepackage{graphicx} % Вставка рисунков

\newcommand{\img}[3] {
	\begin{figure}[H]
		\center{\includegraphics[height=#1]{img/#2}}
		\caption{#3}
		\label{img:#2}
	\end{figure}
}

\usepackage[justification=centering]{caption} % Настройка подписей float объектов

\usepackage[unicode,pdftex]{hyperref} % Ссылки в pdf
\hypersetup{hidelinks}

\usepackage{csvsimple}
\usepackage{float}

%\usepackage[shortcuts]{extdash}

\counterwithin{figure}{section}
\counterwithin{table}{section}
%\numberwithin{equation}{section}


\lstdefinestyle{C++}{ % Опеределение стиля 
	language=C++,
	backgroundcolor=\color{white},
	basicstyle=\footnotesize\ttfamily,
	keywordstyle=\color{blue},
	stringstyle=\color{red},
	commentstyle=\color{gray},
	numbers=left,
	numberstyle=\tiny,
	stepnumber=1,
	numbersep=5pt,
	frame=single,
	tabsize=4,
	captionpos=t,
	breaklines=true,
	breakatwhitespace=true
}
\lstset{extendedchars=true, texcl=true}

\usepackage{makecell}
\usepackage{longtable}

\addto\captionsrussian{\renewcommand{\refname}{Список использованных источников}}

\renewcommand{\labelitemi}{---}
\usepackage{multirow}
\makeatletter
\def\@biblabel#1{#1. }
\makeatother

\lstset{
	literate=
	{а}{{\selectfont\char224}}1
	{б}{{\selectfont\char225}}1
	{в}{{\selectfont\char226}}1
	{г}{{\selectfont\char227}}1
	{д}{{\selectfont\char228}}1
	{е}{{\selectfont\char229}}1
	{ё}{{\"e}}1
	{ж}{{\selectfont\char230}}1
	{з}{{\selectfont\char231}}1
	{и}{{\selectfont\char232}}1
	{й}{{\selectfont\char233}}1
	{к}{{\selectfont\char234}}1
	{л}{{\selectfont\char235}}1
	{м}{{\selectfont\char236}}1
	{н}{{\selectfont\char237}}1
	{о}{{\selectfont\char238}}1
	{п}{{\selectfont\char239}}1
	{р}{{\selectfont\char240}}1
	{с}{{\selectfont\char241}}1
	{т}{{\selectfont\char242}}1
	{у}{{\selectfont\char243}}1
	{ф}{{\selectfont\char244}}1
	{х}{{\selectfont\char245}}1
	{ц}{{\selectfont\char246}}1
	{ч}{{\selectfont\char247}}1
	{ш}{{\selectfont\char248}}1
	{щ}{{\selectfont\char249}}1
	{ъ}{{\selectfont\char250}}1
	{ы}{{\selectfont\char251}}1
	{ь}{{\selectfont\char252}}1
	{э}{{\selectfont\char253}}1
	{ю}{{\selectfont\char254}}1
	{я}{{\selectfont\char255}}1
	{А}{{\selectfont\char192}}1
	{Б}{{\selectfont\char193}}1
	{В}{{\selectfont\char194}}1
	{Г}{{\selectfont\char195}}1
	{Д}{{\selectfont\char196}}1
	{Е}{{\selectfont\char197}}1
	{Ё}{{\"E}}1
	{Ж}{{\selectfont\char198}}1
	{З}{{\selectfont\char199}}1
	{И}{{\selectfont\char200}}1
	{Й}{{\selectfont\char201}}1
	{К}{{\selectfont\char202}}1
	{Л}{{\selectfont\char203}}1
	{М}{{\selectfont\char204}}1
	{Н}{{\selectfont\char205}}1
	{О}{{\selectfont\char206}}1
	{П}{{\selectfont\char207}}1
	{Р}{{\selectfont\char208}}1
	{С}{{\selectfont\char209}}1
	{Т}{{\selectfont\char210}}1
	{У}{{\selectfont\char211}}1
	{Ф}{{\selectfont\char212}}1
	{Х}{{\selectfont\char213}}1
	{Ц}{{\selectfont\char214}}1
	{Ч}{{\selectfont\char215}}1
	{Ш}{{\selectfont\char216}}1
	{Щ}{{\selectfont\char217}}1
	{Ъ}{{\selectfont\char218}}1
	{Ы}{{\selectfont\char219}}1
	{Ь}{{\selectfont\char220}}1
	{Э}{{\selectfont\char221}}1
	{Ю}{{\selectfont\char222}}1
	{Я}{{\selectfont\char223}}1
}

\usepackage{wrapfig}
